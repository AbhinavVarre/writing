\documentclass[11pt]{article}

\usepackage{fullpage,times,hyperref,graphicx}

% The next line is the title of your paper. Update it.
\title{The Definition of a System: Computational or Otherwise} 
% The next line is the author information for your paper. Update it.
\author{Abhinav Varre}
% The next line is the timestamp for your paper. DO NOT CHANGE IT.
\date{\today}

% The next line controls the formatting of your references. DO NOT CHANGE IT.
\bibliographystyle{plain} 

\begin{document}
\maketitle

\section{Introduction}
The definition of a system has been in contention as the English Language grows and evolves. Domains ranging from biology and mathematics to computation and engineering display fundamentally different understandings and interpretations of what one would consider the ``definition''  of a system. Therefore, before one goes about attempting to synthesize and develop a definition for a system, it is important to elucidate the context in which the definition is being applied. 
To begin, this essay will attempt to address the definition of systems from a general perspective and then narrow this definition down so that it aligns with a definition for a ``computational system.'' The literature on this topic is diverse and varies from domain to domain, but this essay will take three separate sources into consideration in order to supplement the argument for the definition.

\section{General System}
What is a system? In the most general, abstract sense, what constitutes what one may consider a system such that the semantic definition can be applied across a number of domains? Russel Ackoff, in his paper ``Systems Thinking and Systems''  lists four attributes of a system that form a holistic definition:
\begin{enumerate}
    \item A ``system is a whole containing two or more parts.''~\cite{ackoff:1994}
    \item Each of these parts ``can affect the performance or properties of the whole.''~\cite{ackoff:1994}
    \item ``No subgroup of [the parts] can have an independent effect on the whole.''~\cite{ackoff:1994}
    \item A system ``cannot be divided into independent parts or subgroups of parts.''~\cite{ackoff:1994}
\end{enumerate}
The word ``part'' as Ackoff uses it is functionally defined as an independent component of the overall system for the purposes of this paper. In any case, the fundamental definition of the word ``system'' in this essay places a heavy emphasis on the idea of a ``whole'' being comprised of a number of individual parts. However,  this definition is incomplete in the sense that it implies that the meaning of the system is derived from the aggregate functions of its parts - or in more colloquial terms, the sum of its parts. 
\subsection{Interaction between Components}
A more complete description would have to encapsulate a very important property of a system - a system is formed by the interaction of its individual components, rather than just a sum of these components. In other words, if one were to lay out all the components of a system in a discrete manner, the system ceases to exist. The individual components must interact with one another in order to satisfy the definition. Ackoff describes this when explaining that ``a system is whole that cannot be divided into independent parts.''~\cite{ackoff:1994} However, it is important to note that the interaction between the parts is what constitutes a system, rather than all of the components being part of the same organizational structure.

\section{Computational Systems}
At this point, the abstract definition of the word system that has been developed is a collection of components or parts that interact with one another. The next section of this essay focuses on narrowing this definition in order to better represent the domain of computation. The handbook ``Hints for Computer Systems Design''~\cite{lampson:1983} by Butler W. Lampson expands upon the idea of how a computational system is comprised of ``interfaces,'' which can be considered systems themselves based on the definition outlined earlier. In the paper, Lampson describes one of the integral attributes of a computer system -  the function that the system provides~\cite{lampson:1983}. Lampson provides a number of axioms to adhere to when designing a computer system, but one of the most important ones is to decide on a solid purview of the functionality that a computer system should offer.~\cite{lampson:1983} This will serve as one of the distinguishing factors from the general definition of a system to one aligned towards computational systems.
\subsection{Predetermined Functionality}
A computational system is designed to, and performs in order to, provide a certain functionality or range of functionality. This definition would not apply to the abstract definition of a system - the domain of biology provides a prime example. An ecosystem is not designed to provide a certain function. It is simply a distinction that is drawn for purposes of classification. Therefore, a computational system is a system that was designed and is used to implement some sort of predetermined functionality.

\section{Computational Systems - Components}
At this point, the definition of a computer system has been extended to include the role functionality plays. However, there is one more important distinction necessary to supplement and complete the definition of a computational system. A relevant example to illustrate this is the paper ``Organizational Structure''~\cite{ahmady:2016} by Ahmady et. al. 
\subsection{The Designation of Components}
Ahmady describes that one of the qualities of ``systemic thinking'' is organizational structure. Organizational structure is defined as ``a set of methods dividing the task to determining duties and coordinates them.''~\cite{ahmady:2016} This highlights a key aspect of computational systems - the explicit designation of every component in the system. This is another specialized attribute of a computational system. Every component is designed to function and interact in a manner consistent with executing the function of the system, and nothing more. 
\subsection{Extraneous Components}
What this means is that there are no extraneous components, in the sense that any given component exists because it provides an integral function to the overall system. Each component is also carefully chosen to adhere to other constraints, such as efficiency and size. The third attribute that will be added to the working definition is that all of the components of a computational system must be necessary to the functioning of the system, or its optimization. There can be no extraneous components in order for a system to be computational. Nothing in the system is arbitrary - every component plays a role and it matters. 

\section{Computational Systems - Computation}
The definition for a system is now narrow enough to specify computational systems. However, the definition still does not address the computational aspect of the system. Therefore, the term computation must be defined in order to comprehend the entirety of the phrase. Computation, in its simplest form, refers to some sort of calculation. The type of calculation can vary, but common examples include arithmetic and logical - both of which are applicable for a computational system. Therefore, the final stipulation to add for the definition of a computational system is that it must undertake some sort of computation, as defined above.

\section{Rubric Design}
\subsection{Requirements for a General System}
In order for something to be defined as a system, it must satisfy two simple requirements. 
\begin{itemize}
    \item The system must be comprised of components. 
    \item These components must interact with one another.
\end{itemize}
These requisites, while seemingly trivial, encompass how this paper defines a system: A whole comprised of multiple parts that interact with one another. It is important that this definition is broad because it must encompass all systems in all domains. For instance, at first glance it might seem prudent to add the specification that all systems must produce something or function in a certain manner - however, this would exclude certain systems, such as an ecosystem. 
\subsection{Approaching a Computational System}
From here, the rubric can be made more specific by adding three more specifications to approach the working definition of a computational system:
\begin{itemize}
    \item The system must perform a predetermined function.
    \item All components must play a role.
    \item There cannot be any extraneous components in the system.
\end{itemize}
These specifications reflect the points in the earlier definition of a computational system. These requirements ensure that only a system that was designed with a specific functionality in mind can qualify - however, this rubric is still somewhat general. For instance, a pulley/lever system designed to move an object would qualify, which would certainly not be computational. Therefore, the rubric must have further requirements to ensure that the system is computational. 
\subsection{Adding Computation}
What the rubric requires is to incorporate the concept of computation - whether it be logical, mathematical, or otherwise. 
\begin{itemize}
    \item The system must compute something in some manner.
\end{itemize}
Adding the stipulation that the system must compute something ensures the appropriate amount of exclusivity in the rubric. Therefore, this rubric, by definition, appropriately addresses both parts of a “computational system.” The rubric in its entirety is outlined below:
\begin{enumerate}
    \item A system must be comprised of components. 
    \item These components must interact with one another.
    \item The system must perform a predetermined function.
    \item Every component must play a role.
    \item There cannot be any extraneous components in the system.
    \item The system must perform some form of computation.
\end{enumerate}

\section{Application of the Rubric}
Now that the rubric has been established, it can be applied to determine whether or not the given exhibits are computational or not.
\subsection{Internal Combustion Engine}
First is the internal combustion engine. It is fairly clear that this artifact satisfies the definition for a general system - the engine is comprised of components, such as the crankshaft, piston, valves, etc. Furthermore, these components interact with one another. The following set of requisites is also fulfilled. The internal combustion engine performs a predetermined function, which simply put is to move or propel what the engine is supposed to power. Every component in the system plays a role - there are no extraneous components that have no role in the function of the system. At this point, the internal combustion engine has passed all of the criteria barring one - the requisite of computation. This system does not perform any variation of computation in order to function. Therefore, the internal combustion engine does not qualify as a computational system. 
\subsection{The Human Brain}
The second artifact is the human brain. This artifact passes the requirement of being comprised of components, such as the cerebrum and cerebellum, and the requirement of these components interacting with one another. Additionally, every component plays a role, and there are no extraneous components in the system. At this level, the only requirement that remains in question is the requirement that the system must perform a predetermined function. This becomes slightly complex - is the function of the brain predetermined? One could argue that this is not the case, because no one person or organization set the directive for the functions a brian must perform - but this argument can easily develop into citing theological and moral perspectives, which is outside the scope of this essay. For the purposes of this essay, because the functions of the brain can be explicitly denoted and predicted in a universal manner (excluding deformities or irregulairites), they will be classified as predetermined. Therefore, the only stipulation remaining is that the brain must perform some sort of computation - which it certainly does, demonstrated by its capacity to perform trillions of calculations in seconds. Therefore, the human brain qualifies as a computational system. 
\subsection{World wide web}
The final artifact is the world wide web. This system is comprised of components such as a server and a client browser, and within these systems, a number of other components, such as hierarchical systems and group - talk ~\cite{berners:1989}. These components all interact with one another - this is done by design, which satisfies the requirement of having a predetermined function. Every component plays a role, and there are no extraneous components because each part of the system was designed with a purpose in mind. Finally, the system also performs the computation in order to function, as is the nature of any system that involves transporting and organizing data. Therefore, the world wide web qualifies as a computational system.

% The following line says where to find your bibliography database.
\bibliography{term-paper-template}
\end{document}